\documentclass[hidelinks]{report}
\usepackage{graphicx} % Required for inserting images
\usepackage{amsmath}
\usepackage{siunitx}
\usepackage{placeins}
\usepackage{tikz}
\usetikzlibrary{automata, arrows, positioning}
\usepackage{float}
\usepackage{hyperref}
\usepackage{cite}
\usepackage[utf8]{inputenc}
\usepackage[T1]{fontenc}
\usepackage{xcolor,graphicx}
\setcounter{secnumdepth}{0}
\usepackage{titlesec}
\usepackage[left=1.5in,top=1in,bottom=1in,right=1in]{geometry}
\usepackage{rotating}
\usepackage{subcaption}
\usepackage{lipsum}
\usepackage{fancyhdr}
\usepackage{mathptmx}
\usepackage{listings}
\usepackage{booktabs}
\usepackage{caption}
\usepackage{subcaption}
\usepackage{adjustbox}
\usepackage{enumitem}
\usepackage{setspace}
\usepackage{multirow}
\usepackage{amsmath}
\usepackage{algorithm}
\usepackage{algpseudocode}
\usepackage{tikz}
\setcounter{tocdepth}{4}
\usepackage{amsfonts}
\usepackage{fvextra}


\DefineVerbatimEnvironment{MyVerbatim}{Verbatim}{breaklines=true}
\usepackage{pgfplots}
\pgfplotsset{compat=1.17}
\newcommand{\mynote}[2]{\fbox{\bfseries\sffamily\scriptsize{#1}} {\small\textsf{\emph{#2}}}}

\newcommand{\hieplnc}[1]{\textcolor{red}{\mynote{hieplnc}{#1}}}

\newcommand{\sontg}[1]{\textcolor{blue}{\mynote{sontg}{#1}}}

\definecolor{blue}{RGB}{31,56,100}

\usepackage{lipsum}% http://ctan.org/pkg/lipsum
\makeatletter
\def\@makechapterhead#1{%
  {
  \parindent \z@ \raggedright \normalfont   
    
    \ifnum \c@secnumdepth >\m@ne
        \huge\bfseries \thechapter.\ % <-- Chapter # (without "Chapter")
    \fi
    \interlinepenalty\@M
    #1\par\nobreak% <------------------ Chapter title
    \vskip 40\p@% <------------------ Space between chapter title and first paragraph
  }}
\makeatother


% Redefine the \thesection and \thesubsection representations
\renewcommand{\thesection}{\arabic{chapter}.\arabic{section}}
\renewcommand{\thesubsection}{\thesection.\arabic{subsection}}
\renewcommand{\thesubsubsection}{\thesubsection.\arabic{subsubsection}}

% Define a new counter for subsections
\newcounter{subsecindex}[section]
\renewcommand{\thesubsecindex}{\thesubsection%
  \ifnum\value{subsecindex}>0
    .\arabic{subsecindex}%
  \fi
}

% Redefine the \section command to include the index
\let\oldsection\section
\renewcommand{\section}[1]{%
  \setcounter{subsecindex}{0} % Reset subsection counter for each section
  \refstepcounter{section}%
  \oldsection{\thesection\hspace{0.5em}#1}%
}

% Redefine the \subsection command to include the index
\let\oldsubsection\subsection
\renewcommand{\subsection}[1]{%
  \refstepcounter{subsection}%
  \oldsubsection{\thesubsecindex\hspace{0.5em}#1}%
}

% Redefine the \subsubsection command to include the index
\let\oldsubsubsection\subsubsection
\renewcommand{\subsubsection}[1]{%
  \refstepcounter{subsubsection}%
  \oldsubsubsection{\thesubsecindex\hspace{0.5em}#1}%
}

\titleformat{\section}
  {\normalfont\LARGE\bfseries} % Adjust \Large to any size you prefer
  {\thesection}{3em}{}
\titleformat{\subsection}
  {\normalfont\Large\bfseries} % Adjust \Large to any size you prefer
  {\thesubsection}{3em}{}
\titleformat{\subsubsection}
  {\normalfont\Large\bfseries} % Adjust \Large to any size you prefer
  {\thesubsubsection}{3em}{}

\begin{document}
\pagenumbering{gobble}

\pdfbookmark[0]{Main Title}{maintitle}
\begin{titlepage}
    \begin{tikzpicture}[remember picture,overlay,inner sep=0,outer sep=0]
        \draw[black!70!black,line width=1.5pt]
            ([xshift=-0.65in,yshift=-1cm]current page.north east) coordinate (A) -- % Adjusted x-shift and y-shift
            ([xshift=0.65in,yshift=-1cm]current page.north west) coordinate (B) -- % Adjusted x-shift and y-shift
            ([xshift=0.65in,yshift=1cm]current page.south west) coordinate (C) -- % Adjusted x-shift and y-shift
            ([xshift=-0.65in,yshift=1cm]current page.south east) -- % Adjusted x-shift
            cycle;
    \end{tikzpicture}

    \begin{center}
    \begin{figure}
        \centering
        \huge \uppercase{university of science and technology of hanoi} \\ [1.5 cm]
    
        \filleft
        \includegraphics[width=0.7\linewidth]{images/usth.png}
    \end{figure}
    
    \textsc{\Large }\\[1cm]
    {\huge \bfseries \uppercase{LABWORK'S REPORT}}\\[1cm]

    {\large \bfseries 2440053 - Dao Xuan Quy  } \\ [0.5cm]
    {\huge \bfseries \uppercase{Advance programming for HPC}}\\[1cm]
    
    % Title
    \rule{\linewidth}{0.3mm} \\[0.4cm]
    { \Huge \bfseries\color{blue}  Labwork 2: Get to know your GPU}
    \rule{\linewidth}{0.3mm} \\[0.7cm]
    
    \large Academic Year: 2024-2026
    \end{center}

\end{titlepage}

\newpage
\pagenumbering{roman}
\noindent \Large \tableofcontents

\newpage
\listoffigures

\newpage
\listoftables

\thispagestyle{empty}
\newpage

\chapter{Introduction}
\pagenumbering{arabic}

\hspace{5mm} In this labwork I run some simple stupid command to get to know my GPU better (I guess)
 
\chapter{Implementation \& Results}
\section{Get GPU's Name and ID}
For the model id and name, use the command: \texttt{numba.cuda.detect()}:
\begin{figure}[h!]
    \centering
    \includegraphics[width=0.8\linewidth]{IMG_0080.jpeg}
    \caption{ID and Name of the GPUs}
    \label{fig:placeholder}
\end{figure}

\section{Multiprocessors and Multicores count}
For multiprocessor count, the command is

\texttt{cuda.get\_current\_device().MULTIPROCESSOR\_COUNT}
\begin{figure}[h!]
    \centering
    \includegraphics[width=0.9\linewidth]{IMG_0081.jpeg}
    \caption{Multiprocessor's count}
    \label{fig:placeholder}
\end{figure}

For multicore count, the formula is written as follow:


$mutilcore\_count = (MAX\_THREADS\_PER\_MULTI\_PROCESSOR / WARP\_SIZE) * (MAX\_THREADS\_PER\_MULTI\_PROCESSOR / MAX\_THREADS\_PER\_BLOCK) * MULTIPROCESSOR\_COUNT$

\section{Memory size}

Simple as that, the command for memory size is: 

\texttt{numba.cuda.current\_context().get\_memory\_info()}
This function returns the free memory of the GPU and the total memory of the GPU.

\begin{figure}
    \centering
    \includegraphics[width=0.9\linewidth]{IMG_0084.jpeg}
    \caption{Memory infomation}
    \label{fig:placeholder}
\end{figure}

\chapter{Conclusion}
In this lab, I tried some functions to get to know my GPU better. 
\end{document}